% Created 2022-07-26 Tue 00:05
% Intended LaTeX compiler: pdflatex
\documentclass[11pt]{article}
\usepackage[utf8]{inputenc}
\usepackage[T1]{fontenc}
\usepackage{graphicx}
\usepackage{longtable}
\usepackage{wrapfig}
\usepackage{rotating}
\usepackage[normalem]{ulem}
\usepackage{amsmath}
\usepackage{amssymb}
\usepackage{capt-of}
\usepackage{hyperref}
\date{}
\title{}
\hypersetup{
 pdfauthor={},
 pdftitle={},
 pdfkeywords={},
 pdfsubject={},
 pdfcreator={Emacs 26.3 (Org mode 9.5.4)}, 
 pdflang={English}}
\begin{document}

\tableofcontents



\section{introduction}
\label{sec:orgabdef5d}

\section{Basic idea}
\label{sec:org8c9ae69}

A dungeon crawler set in a modern day office space (with cubicles, cantinas etc.) instead of fighting I need another way of handling engagement.

\section{"Races"}
\label{sec:orgcb8e7fd}

\begin{itemize}
\item Zzzombie - the constantly sleepy coworker, never seems to be quite awake nor have they had a good nights sleep in their life.
\item Gobbo - The coworker who handles mails, you're not sure if they've ever had a shower, but the greasy strands of hair makes you think not.
\item Creeps -
\item Emotional vampire
\item Ghost
\item Troglodytes
\end{itemize}


\section{Classes}
\label{sec:org6296986}

\begin{itemize}
\item Financial
\item IT
\item Assistant
\item Intern
\item Secretary
\item Janitor
\item Boss
\end{itemize}

\section{Stats}
\label{sec:orgf0513aa}

Stats in an RPG is like\ldots{} something\ldots{} really\ldots{} hmm (ed: come back with a zinger here). Usually the stats in a traditional roguelike might be something akin to - Constitution, Strength, Wisdom, Intelligence etc. basically if you've seen a rulebook for a TTRPG you can probably recognize most of these AND THEY WORK! that's why people use them, they really REALLY work. I'm not choosing differently because I think I can do better, but because, well since the game is already set in a modern office environment, there's little reason to not have some weird quirk with the stats again.

\subsection{List of stats to consider:}
\label{sec:org068ab22}
\begin{itemize}
\item Neuroticism: how panicked and alert a player is about their surrounding.
\item Openness: how open they can allow themself to be.
\item Reasoning: how good they are at logically deducting things, (maybe reasoning and neuroticism can kindda benefit and be a detriment at the same time)
\item Madness: How mad the character is (remember! you don't have to be mad to work here, but it helps!)
\item Agreeableness: How much other characters like you.
\item Laziness: How much (or little :/ ) energy a character has.
\end{itemize}


I Really REALLY wanted to put "Agility" into this stat list, but not agility as it is traditionally used, but more as in the agile method that is usually claimed by every company under the sun that has too much money and not enough sensible managers/CEOs. Though I feel like that'll be an unnescessary dig at alot of people I have tremendous respect for. (though maybe they don't care what a nobody on the internet does or say\ldots{} I dunno :) )

So that's the N.O.R.M.A.L. stat system (you'd think I'd have picked these words carefully\ldots{} and you'd be right) patent pending yadda yadda.

\section{Code}
\label{sec:org9701cf3}

How neat would it be if we could quit here? alas now we have to try and code this :S 




\begin{verbatim}
from __future__ import annotations
from abc import ABC, abstractmethod
import pygame
import pprint
from pygame.locals import *
import numpy
from collections.abc import Callable
from functools import reduce, cache, wraps, partial
import json
from dataclasses import dataclass, field
import operator
\end{verbatim}

\section{Globals}
\label{sec:orgee6c466}

Global variables are frowned upon by virtually everyone and their mom, which is fair, I don't like them either, but I don't have a better idea.

Global variables are the following
\begin{itemize}
\item cell size, i.e. how large the cell that fills the screen surface should be. Then I can later just divide the screenwidth or screenheight of the drawing destination surface by the cell size to know how big the are is. (so I don't draw outside the screen/surface), this is also where I'm placing the filenames for fonts which I might need/want to change.
\end{itemize}


\begin{verbatim}
CELLSIZE = 32
FONTFILE = "terminal8x8_gs_ro.png"
\end{verbatim}


\section{Positional handling}
\label{sec:orge7cc6b4}

Positional functions all tackles different aspect of onscreen positioning. Things lige moving, checking collisions. multiplying a vector tuple so that it uses the correct 

\subsection{Cell function}
\label{sec:org3daac51}

The cell function handles converting the direction vectors into coordinates that can be used on the screen

\begin{verbatim}
def Cell(cell : (int, int)) -> (int, int):
    return tuple(map(lambda n : n * CELLSIZE, cell))
\end{verbatim}


\subsection{Collision}
\label{sec:orgf07b063}

The collision function gets two position tuples and checks if they are the same, if they are it returns true if not it returns false

\begin{verbatim}
def collision(xy, _xy):
    return xy == _xy
\end{verbatim}


\subsection{Move}
\label{sec:org9997826}

Move function is just meant to take tqwo positional arguments, the current position and the destination, and return a new tuple with the new current position. I believe It \textbf{could} theoretically maybe, potentially handle diagonal movement - ish but this is just you grandmas 4 directional moves.

\begin{verbatim}
def move(xy : (int,int), _xy : (int, int)) -> (int, int):
    acc = []
    for n, _n in zip(xy, _xy):
        if n > _n:
            n = n - 2
        elif n < _n:
            n = n + 2
        acc.append(n)
    return tuple(acc)

\end{verbatim}


\subsection{Vector data struct}
\label{sec:orgab78fa8}

\begin{verbatim}
@dataclass(frozen=True)
class Vector:
    x : int
    y : int
    def __add__(self, other):
        return Vector(self.x + other.x, self.y + other.y)
    def __sub__(self, other):
        return Vector(self.x - other.x, self.y - other.y)
\end{verbatim}


\section{Input}
\label{sec:org466a1c6}

\subsection{Input player movement}
\label{sec:orgcc40204}

for now I just handle the input through a simple function that checks whether or not a valid key has been pressed. if not it returns a (0,0) vector. 

\begin{verbatim}
def getInput(ev):
    inputList = { pygame.K_UP : (0,-1),
                  pygame.K_DOWN : (0,1),
                  pygame.K_LEFT : (-1, 0),
                  pygame.K_RIGHT : (1, 0),
    }
    return inputList.get(ev.key, (0,0))
\end{verbatim}





\subsection{Ressource handler}
\label{sec:orgf832be1}

Another euqually import (or more important input) is the different ressources. For now it is only a config json file and a tilesheet, I'm interested in, but it could expand to more tilesheets, or even premade levels (or templates). The ressources are being loaded into a file class that holds the different information.

\begin{verbatim}
def loadFiles() -> ConfigFile:
    _fontImage = pygame.image.load("terminal8x8_gs_ro.png")
    _fontImage.set_colorkey((0,0,0))
    with open('conf.json', 'r') as _file:
        config = json.load(_file)
    return ConfigFile(_fontImage, config)
\end{verbatim}

\subsubsection{File class}
\label{sec:org86f23f5}

The file class is an immutable data container that holds the information needed for the game to function

\begin{verbatim}
@dataclass(frozen=True)
class ConfigFile:
    _image : pygame.Surface
    _conf : dict
    def image(self) -> pygame.Surface:
        return self._image
    def config(self) -> dict:
        return self._conf
\end{verbatim}



\subsubsection{Tilesheet class}
\label{sec:orgde1f412}

\begin{verbatim}
@dataclass(frozen=True)
class TileSheet:
    _tiles = [pygame.Surface]
    def __init__(self,
                 file : pygame.Surface,
                 width : int,
                 height : int,
                 rows : int,
                 columns : int):
        for x in range(rows):
            for y in range(columns):
                self._tiles.append(
                    pygame.transform.scale(
                        file.subsurface(y * width, x * height, width, height),
                        (CELLSIZE, CELLSIZE)
                    )
                )
\end{verbatim}

\section{Window handler}
\label{sec:org7ab670c}

The window class is where the initializing is going to happen, as well as where the pygame window.


\begin{verbatim}
class Window():
    _size = (int, int)
    _surface = pygame.Surface
    def __init__(self, width, height) -> None:
        self._surface = pygame.display.set_mode((width, height), 0, 32)

        
    def surface(self) -> pygame.Surface:
        return self._surface
\end{verbatim}


\section{View MIGHT BE REMOVED}
\label{sec:org06796e7}

The view holds a surface and a pygame.Rect. The rect is moved around to "slice" a subsurface from the map. 



\begin{verbatim}
class View:
    _camera : pygame.Rect
    _trackingObject = None
    def __init__(self,
                 topx : int,
                 topy : int,
                 cameraWidth : int,
                 cameraHeight : int,
                 trackingObject = None):
        self._camera = pygame.Rect(topx, topy, topx+cameraWidth, topycameraHeight)
        if trackingObject is not None:
            self._trackingObject = trackingObject

    def slice(self, surface : pygame.Surface):
        return surface.Subsurface(self._camera)

    def trackObject(self, surface : pygame.Surface):
        pass

    def checkCenterObject(self):
        pass

    def update(self) -> pygame.Surface:
        pass
\end{verbatim}

The view function takes the relevant actor and centers the map on it.

\begin{verbatim}
def View(actor : Actor,
         surface : pygame.Surface,
         view : pygame.Rect) -> pygame.Surface:
    _view = view
    _view.center = actor.currxy()
    return surface.subsurface(_view)
\end{verbatim}


\section{Drawing}
\label{sec:org3060db8}

\subsection{Drawmap}
\label{sec:orgf37f990}
Drawmap function is only called to draw the surface of the static map.

\begin{verbatim}
def drawMap(map : str,
            pos : [(int, int)],
            tiles : [pygame.Surface],
            destination :pygame.Surface):
    _drawingList : [(pygame.Surface,(int,int))] = []
    x = 0
    y = 0
    for ind, c in enumerate(map):
        _drawingList.append((tiles[ord(c)+1], pos[ind]))
    destination.blits(_drawingList)
\end{verbatim}

\subsection{Drawing}
\label{sec:org377e17e}
The drawing function takes one or more characters and draws them to the screen.

\begin{verbatim}
def drawing(chars : str,
            pos : [(int, int)],
            tiles : [pygame.Surface],
            destination :pygame.Surface):
    _drawingList : [(pygame.Surface,(int,int))] = []
    x = 0
    y = 0

    for ind, c in enumerate(chars):
        _drawingList.append((tiles[ord(c)+1], pos[ind]))
    destination.blits(_drawingList)


\end{verbatim}


\section{Map}
\label{sec:orgb1f452d}

the map is for now just a container class for a premade "dungeon", this is to test whether or not the drawing function can handle the sheer drawing calls. AND that it can handle the various characters.

It's supposed to just have a giant, static (more or less static) image of the map.

\begin{verbatim}
class Map:
    _str : str
    _pos : [(int,int)] = []
    _map : pygame.Surface
    def __init__(self, tiles):
        # TEMP map
        self._str = ""
        tempMap = [ "################################",
                    "#       #             #        #",
                    "#       #             #        #",
                    "#       #             #        #",
                    "#       #             #        #",
                    "#                              #",
                    "#                              #",
                    "#           y                  #",
                    "#      Hello                   #",
                    "#       p                      #",
                    "#                              #",
                    "#                              #",
                    "#                              #",
                    "#                              #",
                    "#                              #",
                    "################################"
        ]
        w = len(tempMap[0]) * CELLSIZE
        h = len(tempMap) * CELLSIZE
        self._map = pygame.Surface((w, h))
        
        for i, s in enumerate(tempMap):
            self._str = self._str + s
            for string_i, _ in enumerate(s):
                self._pos.append((string_i*CELLSIZE, i*CELLSIZE))
        drawMap(self._str, self._pos, tiles, self._map)

    def map(self):
        return self._map
\end{verbatim}


\section{Actor list}
\label{sec:org3d99174}

The actor list is where a list of actors are being created.

\begin{verbatim}
def makeActors(amount : int, playerIndex : int) -> [Player]:
    pass
\end{verbatim}


\section{Game Loop function}
\label{sec:org72f4636}

The Game loop is where the structure of the game is at.

\begin{verbatim}
def GameLoop(window, _map, tiles):
    running = True
    player = Actor('@',(32,32), (32,32))
    currVec = player.currxy()
    nxtVec = player.nxtxy()
    log = []
    cl = pygame.time.Clock()
    while(running):
        movVec = (0,0)
        for event in pygame.event.get():
            match event.type:
                case pygame.QUIT:
                    running = False
                case pygame.KEYDOWN:
                    movVec = getInput(event)
        if player.arrived():
            nxtVec = tuple(map(lambda n, _n: n + ( _n * CELLSIZE), nxtVec, movVec))           
        currVec = move(player.currxy(), nxtVec)
        player = Actor( '@',currVec, nxtVec)
        window.surface().blit(_map.map(), (0,0))
        drawing(str(player), [player.currxy()], tiles._tiles, window.surface())
        pygame.display.flip()
        log.append("current vector : " + str(currVec) + "nxtVector : " + str(nxtVec) + "has player arrived: " + str(player.arrived()))
        cl.tick_busy_loop(60)
    pprint.pprint(_map.map())
\end{verbatim}

\section{Texthandling}
\label{sec:orge73a0bc}



\section{Actor}
\label{sec:org3ae993f}

The player class is, for now just a place holder, keeping the player char (literally a char value), position and that's it. It does also use a couple of standard operators that I've overloaded to return just a string.

\begin{verbatim}
@dataclass(frozen=True)
class Actor:
    _c : chr = field(init=True)
    xy : (int, int) = field(init=True)
    _xy : (int, int) = field(init=True)
    def arrived(self) -> bool:
        return self.xy == self._xy

    def currxy(self):
        return self.xy

    def nxtxy(self):
        return self._xy

    def __repr__(self):
        return self._c
\end{verbatim}


\section{Update Actors}
\label{sec:org394d52f}

The update actor function is going to map onto the list of actors in the game.

\begin{verbatim}
def updateActor(lstOfActors : [Actor]) -> [Actor]:
    # retLst = []
    # #check collision
    # lstCollision
    # for nxt in lstOfActors:
    #     lstCollision.append(nxt.
    # for actor in lstOfActors:
    #     currVec = move(actor.currxy(), actor.nxtxy())
    pass
\end{verbatim}



\section{Main}
\label{sec:orge58cd1c}

In the main function I initialize the different components, like the window, the gameloop function, etc.

\begin{verbatim}
def main():
    pygame.init()
    pygame.font.init()
    # -------
    _files = loadFiles()
    _tiles = TileSheet(_files.image(), 8, 8, 16, 16)
    window = Window(800, 600)
    map = Map(_tiles._tiles)
    GameLoop(window, map, _tiles)
    # -------
    pygame.quit()
\end{verbatim}



\begin{verbatim}
if __name__=="__main__":
    main()
\end{verbatim}
\end{document}